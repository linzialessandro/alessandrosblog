\documentclass[11pt,a4paper,oneside]{memoir}

% Typography + fonts
\usepackage{fontspec}
\usepackage[T1]{fontenc}
\usepackage{microtype}

% Font fallback: use Libertinus if available, otherwise TeX Gyre Pagella.
\IfFontExistsTF{Libertinus Serif}{
  \setmainfont{Libertinus Serif}
  \setsansfont{Libertinus Sans}
  \setmonofont{Libertinus Mono}
}{
  \setmainfont{TeX Gyre Pagella}
  \setsansfont{TeX Gyre Heros}
  \setmonofont{TeX Gyre Cursor}
}

% Layout (memoir-native)
\setlrmarginsandblock{1.15in}{1.15in}{*}
\setulmarginsandblock{1.0in}{1.0in}{*}
\checkandfixthelayout

% No headers/footers/page numbers
\pagestyle{empty}

% Paragraph style
\setlength{\parindent}{0pt}
\setlength{\parskip}{0.55\baselineskip}

% Pandoc helpers
\usepackage{xcolor}
\usepackage[hidelinks]{hyperref}
\providecommand{\tightlist}{%
  \setlength{\itemsep}{0pt}\setlength{\parskip}{0pt}}

\begin{document}

{\Large\bfseries How People Actually Use AI Agents (It’s Mostly Cognitive Work)\par}
\vspace{1.25em}

There's a popular story about AI agents as ``digital concierges'' that book hotels, schedule meetings, and run errands. Useful, sure---but also a bit narrow. A recent study from Perplexity and Harvard researchers (released in December 2025) looks at real usage at scale and lands on a different picture: agents are increasingly used as cognitive partners, not just task-runners.

\subsection{Cognitive work dominates}\label{cognitive-work-dominates}

The headline finding is striking: 57\% of agent activity is cognitive work, split between Productivity \& Workflow (36\%) and Learning \& Research (21\%). In other words, a lot of people aren't delegating ``boring chores'' as much as they're delegating the messy middle of knowledge work: synthesizing information, navigating complexity, and turning scattered inputs into decisions.

That maps well to real examples: a professional scanning case studies to extract patterns, or a student using an agent to navigate course material and make it more searchable and explainable. This is less about avoiding work and more about compressing the overhead that normally slows work down.

\subsection{How usage evolves over time}\label{how-usage-evolves-over-time}

Another useful insight is the progression. New users tend to start with low-stakes queries (travel ideas, trivia, entertainment), then shift toward higher-leverage uses once they see what's possible---debugging code, summarizing reports, planning complex workflows, or structuring learning. The study describes this as a ``pull'' toward productivity: once people experience the leverage, they don't fully go back.

\subsection{Who sticks with agents}\label{who-sticks-with-agents}

Adoption isn't uniform across professions. Digital technologists lead in volume (30\% of queries), but knowledge-intensive fields like Marketing, Sales, Management, and Entrepreneurship show high ``stickiness''---usage intensity that outpaces their adoption share once they integrate agents into daily workflow.

Context matters too: personal use accounts for 55\% of queries, followed by professional (30\%) and educational (16\%). That mix is a reminder that ``agent value'' isn't only about enterprise automation; it's also about making everyday life and learning less cognitively expensive.

\subsection{Why this matters}\label{why-this-matters}

The most interesting implication is that the near-term impact of agents might be about scaling cognition rather than replacing labor. If agents primarily accelerate synthesis, learning, and workflow setup, then the economic shift looks like ``hybrid intelligence'': people + tools that extend attention, memory, and speed.


\end{document}
