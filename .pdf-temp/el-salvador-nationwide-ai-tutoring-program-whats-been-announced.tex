\documentclass[11pt,a4paper,oneside]{memoir}

% Typography + fonts
\usepackage{fontspec}
\usepackage[T1]{fontenc}
\usepackage{microtype}

% Font fallback: use Libertinus if available, otherwise TeX Gyre Pagella.
\IfFontExistsTF{Libertinus Serif}{
  \setmainfont{Libertinus Serif}
  \setsansfont{Libertinus Sans}
  \setmonofont{Libertinus Mono}
}{
  \setmainfont{TeX Gyre Pagella}
  \setsansfont{TeX Gyre Heros}
  \setmonofont{TeX Gyre Cursor}
}

% Layout (memoir-native)
\setlrmarginsandblock{1.15in}{1.15in}{*}
\setulmarginsandblock{1.0in}{1.0in}{*}
\checkandfixthelayout

% No headers/footers/page numbers
\pagestyle{empty}

% Paragraph style
\setlength{\parindent}{0pt}
\setlength{\parskip}{0.55\baselineskip}

% Pandoc helpers
\usepackage{xcolor}
\usepackage[hidelinks]{hyperref}
\providecommand{\tightlist}{%
  \setlength{\itemsep}{0pt}\setlength{\parskip}{0pt}}

\begin{document}

{\Large\bfseries El Salvador’s Nationwide AI Tutoring Program: What’s Been Announced\par}
\vspace{1.25em}

A notable education announcement landed this week: El Salvador and xAI say they're launching what they describe as the first nationwide AI-powered education program. The plan is to deploy Grok across more than 5,000 public schools over the next two years, with the stated goal of delivering personalized tutoring to over one million students and support for teachers.

\subsection{What the program aims to do}\label{what-the-program-aims-to-do}

The core idea is an ``adaptive tutor'' that aligns with the national curriculum and adjusts to each student's pace and current level. If implemented well, that kind of personalization could matter most in the long tail of learning: students who move faster than the class average, students who need extra repetition, and students in settings where teacher-to-student ratios make 1:1 help hard.

The announcement also emphasizes that the tool is meant to work alongside educators, not in isolation---positioning it as something that can support teachers with explanations, practice, and targeted reinforcement rather than replace classroom instruction.

\subsection{Why this rollout is unusual}\label{why-this-rollout-is-unusual}

Plenty of schools experiment with AI tutors, but doing it at national scale changes the problem. It turns ``does this help in a pilot?'' into questions like: how do you ensure curriculum fit, consistency across regions, equitable access, and safe defaults for minors? A rollout to 5,000+ schools forces those operational and governance issues to become first-class engineering requirements.

\subsection{What to watch next}\label{what-to-watch-next}

The big unknowns are in the details: evaluation methods, guardrails, teacher training, how student data is handled, and how the system behaves under real classroom constraints (limited connectivity, device availability, different grade levels, and language needs). If El Salvador publishes frameworks or measurement outcomes from this deployment, those could become a reference point for other governments exploring similar programs.


\end{document}
