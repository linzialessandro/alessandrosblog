\documentclass[11pt,a4paper,oneside]{memoir}

% Typography + fonts
\usepackage{fontspec}
\usepackage[T1]{fontenc}
\usepackage{microtype}

% Font fallback: use Libertinus if available, otherwise TeX Gyre Pagella.
\IfFontExistsTF{Libertinus Serif}{
  \setmainfont{Libertinus Serif}
  \setsansfont{Libertinus Sans}
  \setmonofont{Libertinus Mono}
}{
  \setmainfont{TeX Gyre Pagella}
  \setsansfont{TeX Gyre Heros}
  \setmonofont{TeX Gyre Cursor}
}

% Layout (memoir-native)
\setlrmarginsandblock{1.15in}{1.15in}{*}
\setulmarginsandblock{1.0in}{1.0in}{*}
\checkandfixthelayout

% No headers/footers/page numbers
\pagestyle{empty}

% Paragraph style
\setlength{\parindent}{0pt}
\setlength{\parskip}{0.55\baselineskip}

% Pandoc helpers
\usepackage{xcolor}
\usepackage[hidelinks]{hyperref}
\providecommand{\tightlist}{%
  \setlength{\itemsep}{0pt}\setlength{\parskip}{0pt}}

\begin{document}

{\Large\bfseries AI Fluency: A Better Way to Work with AI (Beyond Prompt Hacks)\par}
\vspace{1.25em}

There's a whole mini-industry around ``the perfect prompt,'' but most of those tricks decay fast: models change, interfaces change, and the hack stops working. What's more useful is a stable mental model for collaborating with AI across tools and contexts.

Anthropic's free AI Fluency course takes that route. Instead of teaching a bag of prompt hacks, it teaches a framework for working with AI effectively, efficiently, ethically, and safely, built around four core competencies (the ``4Ds''). The course is developed in partnership with academic experts Joseph Feller and Rick Dakan.

\subsection{The 4Ds that make it practical}\label{the-4ds-that-make-it-practical}

The framework is simple enough to remember, but deep enough to apply repeatedly:

\begin{itemize}
\tightlist
\item
  \textbf{Delegation}: Decide what should be done by you, what should be done by the model, and how to split tasks so you don't outsource judgment by accident.
\item
  \textbf{Description}: Communicate intent and constraints clearly (context, audience, format, examples), so the model has something concrete to aim for.
\item
  \textbf{Discernment}: Evaluate outputs critically---both the final result and how the model got there---so you can catch errors, weak logic, and hidden assumptions.
\item
  \textbf{Diligence}: Use AI responsibly: be transparent where needed, stay accountable for what you publish, and consider downstream impacts.
\end{itemize}

\subsection{Why this approach scales}\label{why-this-approach-scales}

What I like about this is that it's tool-agnostic. The same habits apply whether the UI says Claude, ChatGPT, Grok, or something else---because the bottleneck isn't the brand, it's how well you set up the collaboration and how seriously you verify what comes back.

The course also leans into practice: interactive exercises, real-world projects, and a ``Bad Prompt Makeover'' style activity that forces you to notice why vague requests create vague results. There's also a completion certificate, which is a nice forcing function if finishing courses usually drifts to the bottom of the todo list.

\subsection{What this is really about}\label{what-this-is-really-about}

AI is already reshaping workflows and job roles, but ``better prompting'' isn't the endgame. The people who get leverage will be the ones who can delegate strategically, communicate precisely, evaluate ruthlessly, and stay responsible for outcomes.

You can enroll for free in the course here: \href{https://anthropic.skilljar.com/ai-fluency-framework-foundations}{Anthropic Academy}


\end{document}
