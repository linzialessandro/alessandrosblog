\documentclass[11pt, a4paper, extrafontsizes,openany]{memoir}

% --- Language & Encoding ---
\usepackage{fontspec}
\usepackage{polyglossia}
\setmainlanguage{english}

% --- Typography ---
% FIX: We rely on fontspec's default behavior. 
% By NOT specifying \setmainfont, fontspec automatically loads 
% Latin Modern using internal filenames, avoiding system lookup errors.
\defaultfontfeatures{Ligatures=TeX}

% --- Layout & Design ---
\setlrmarginsandblock{3.5cm}{3.5cm}{*}
\setulmarginsandblock{3.5cm}{3.5cm}{*}
\checkandfixthelayout

\usepackage{xcolor}
\definecolor{accentcolor}{RGB}{41, 128, 185} % A calm blue
\definecolor{darkgray}{RGB}{50, 50, 50}

% --- Links ---
\usepackage[colorlinks=true, linkcolor=accentcolor, urlcolor=accentcolor]{hyperref}
\usepackage{graphicx}

% --- Title Styling ---
\pretitle{\begin{center}\includegraphics[width=4cm]{assets/logo.png}\\[1em]\Huge\bfseries\color{darkgray}}
\posttitle{\par\vskip 0.5em\normalfont\large\textit{A manifesto for a quieter web}\end{center}\vskip 2em}
\preauthor{\begin{center}\large\lineskip 0.5em}
\postauthor{\par\end{center}}
\predate{\begin{center}\large}
\postdate{\par\end{center}}

% --- Section Styling ---
\setsecheadstyle{\Large\bfseries\color{darkgray}}
\setsubsecheadstyle{\large\bfseries\itshape\color{darkgray}}

\begin{document}

% --- Title Page ---
\title{Running My Blog in 2025: Behind the Scenes}
\author{Alessandro Linzi}
\date{December 2025}

\maketitle

\thispagestyle{empty}
\begin{abstract}
    \noindent This is the story of why I stopped scrolling and started building. It is a look at the philosophy of "learning out loud" and how modern AI tools like Google Antigravity and Perplexity helped me construct a privacy-first, open-source haven for ideas in an increasingly noisy digital world.
\end{abstract}

\vfill
\begin{center}
    \small \url{https://alessandrosblog.it.eu.org}
\end{center}
\newpage

% --- Content ---

\chapter*{The Signal and the Noise}

For years, my "learning out loud" was scattered across social timelines—ephemeral bursts of insight sandwiched between ads and algorithmic distractions. By late 2024, I realized that while I was sharing, I wasn't \textit{building}. I was renting space on platforms that incentivized engagement over depth.

I wanted a home. A digital garden where ideas could mature, where mathematics and code could live side-by-side, and where "privacy" wasn't just a toggle in a settings menu, but the default state of existence.

This blog is that home. It is a return to the "small web"—static, fast, and owned entirely by me (and now, by you).

\section*{Building in the Age of AI Agents}

Building a blog from scratch in 2025 is a paradox. On one hand, we have tools of immense complexity; on the other, I wanted a result of extreme simplicity. I didn't want a heavy framework. I wanted pure HTML, CSS, and a JSON file.

To achieve this, I leaned heavily on the AI tools that have defined this year:

\subsection*{Perplexity: The Architect}
Before a single line of code was written, I used \textbf{Perplexity} to challenge my assumptions. I didn't just ask \textit{"how to build a blog"}; I asked it to critique the modern web. We debated the necessity of cookies (conclusion: unnecessary), the weight of JavaScript frameworks (conclusion: often bloated), and the best structure for a flat-file database.

Perplexity helped me design the \texttt{posts.json} schema that powers this site. It acted as a research partner, filtering through the noise of "best practices" to find the timeless ones: semantic HTML and accessible design.

\subsection*{Google Antigravity: The Builder}
When it came time to write the infrastructure, I turned to \textbf{Google Antigravity}. The agentic capabilities of the platform were instrumental in writing the tooling that keeps this blog sane.

I delegated the creation of \texttt{blogq}—my custom Python validator—to an Antigravity agent. I described the rules: \textit{"No slugs with capital letters, no tracking pixels, force secure links, and validate the JSON schema strictly."} The agent wrote the rigorous test suite that now runs before every commit. It allowed me to focus on the \textit{content} of the proofs and posts, knowing the \textit{structure} was being guarded by an automated engineer we built together.

\section*{The Philosophy: Learning Out Loud}

The core mission of this space is to demystify complexity. Whether it's a formal proof in Isabelle/HOL or a breakdown of the latest sustainability algorithms in AI, the goal is to show the work.

\begin{itemize}
    \item \textbf{Privacy First:} You will find no cookie banners here because there are no cookies. I do not track you. The only metric that matters is whether you found something useful.
    \item \textbf{Performance as Respect:} The site loads instantly because your time is valuable. There are no build steps between me writing and you reading, just a raw sync of text.
    \item \textbf{Open Source by Default:} This entire project is open source. The engine that runs this blog is available for you to inspect, critique, and steal.
\end{itemize}

\section*{A Hope for the Future}

My hope is that this blog serves as more than just a repository of my thoughts. I want it to be a template for yours.

We are moving toward a web where personal publishing is reclaimed from walled gardens. By making this project open source, I invite you to clone it. Use the \texttt{blogq} tool to validate your own ideas. Improve the \texttt{build-pdfs.mjs} script to generate better documents. 

If you find a typo in a proof, fix it. If you have a better way to explain a concept, submit a draft. This is a living document of my learning journey, but it is hosted on an open platform so that it might become part of yours.

Welcome to the behind-the-scenes. 

\vspace{2em}
\noindent\textit{— Alessandro}

\end{document}
